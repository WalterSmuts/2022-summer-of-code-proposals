\documentclass{article}
\usepackage[utf8]{inputenc}
\usepackage[margin=0.75in]{geometry}
\usepackage{hyperref}
\usepackage{float}

\hypersetup{urlcolor=blue,colorlinks=true,linktocpage}

\newcommand\ytl[2]{
\parbox[b]{10em}{\hfill{\bfseries\sffamily
#1}~$\cdots\cdots$~}\makebox[0pt][c]{$\bullet$}\vrule\quad \parbox[c]{8cm}{\vspace{3pt}\raggedright\sffamily #2.\\[3pt]}\\[-3pt]}

\begin{document}
\begin{center}


\LARGE{\textbf{Haskell.org}} \\
\vspace{1em}
\Large{HLS - hlint editor end-to-end integration} \\
\vspace{1em}
\normalsize\textbf{Walter Smuts} \\
\normalsize{smuts.walter@gmail.com} \\
\vspace{1em}
\normalsize{Mentor: Michael Peyton Jones} \\
\vspace{1em}
\normalsize{Google Summer of Code} \\

\end{center}
\begin{normalsize}

\section{Project Description:}

Language servers are an attempt at solving the problem of each text editor having
to implement all the IDE-features, like autocomplete, syntax highlighting etc, for
each language it want's to support. If N editors implements these features for M
languages we have N x M implementations. If we instead implement a language server
for each language and they communicate over a common protocol with each editor
then we've successfully reduced the number of projects to N + M. Great success!

Haskell.org maintains an LSP server for haskell called Haskell Language Server
(HLS). The most common features are implemented but some useful ones are still
remaining. The one relevant to this project is called
\href{https://microsoft.github.io/language-server-protocol/specifications/lsp/3.17/specification/#textEdit}{AnnotatedTextEdit}.
This feature allows an LSP client to make annotated edits and display the edit
with the annotation in the editor and optionally prompt the user to accept or
reject the edit.

The use-case this enables is to be able to run the
\href{https://github.com/ndmitchell/hlint}{hlint tool}, a linting tool similar to
clippy for rust, from within a text editor. The editor can then prompt the user
for each edit with an annotation explaining why this edit is done, giving the user
the ability to accept or reject the edit.

In order to get the entire workflow to work this will most likely require changes
to HLS, hlint and a text editor. My editor of choice is vim/nvim, which I expect would
mean changes to the LSP plugin to support this use-case fully. I'd be open to
implement this for other common editors too if it's desired.

If the project goes faster than anticipated I'd be happy to implement other LSP
features for HLS.

\section{Timeline:}

This timeline takes into account that I'd be an absolute beginner to Haskell
programming and moving much slower than someone already fluent in the language. My
personal interests in pursuing this project is to finally be forced to get my
hands dirty with Haskell and not just faf around with theory and philosophy.

\begin{table}[H]
\centering
\begin{minipage}[t]{.7\linewidth}
\rule{\linewidth}{1pt}
\ytl{Week 1 - 2}{Get consensus on desired workflow and start familiarizing myself with all the different components}
\ytl{Week 3 - 6}{Hack a POC together}
\ytl{Week 7 - 8}{Compile a set of reviewable/maintainable patches for each project}
\ytl{Week 9 - 10}{Code review process and merge PRs}
\ytl{Week 11 - 12}{Buffer Time - Things always take longer than expected}
\bigskip
\rule{\linewidth}{1pt}%
\end{minipage}%
\end{table}

\section{Deliverables:}

The intended overall deliverable to the Haskell community is more engagement with
the hlint tool and a better understanding of idiomatic Haskell. Breaking up the
deliverables into concrete chunks results in this list:

\begin{itemize}
\setlength\itemsep{-0.1em}
\item Implementation of the AnnotatedTextEdit feature in HLS
\item Implementation of hlint communication with LSP
\item Atleast one end-to-end working workflow of editor integrated hlint
\item Lowering the barrier to using tools like hlint
\item A Haskell community more in-tune with idiomatic Haskell
\item Maybe some more LSP features
\end{itemize}

\section{Contributor Info:}

I am Walter Smuts, an Electrical and Computer Engineering graduate from the
University of Cape Town. After I finished my studies I worked for Amazon Web
Services in Cape Town. I worked on a micro-services team on the EC2 control plane
for a year, mostly maintaining software written in Java. Thereafter I joined a
team doing new feature development on the AWS EC2 hypervisor working primarily in
C and Rust. I spent two years here and really enjoyed the work and my team.
Although I  was happy, I felt like I needed to do some more traveling while I'm
young before getting stuck in the corporate world.

I left my job with the idea to make some room to travel and think about my next
venture. Working on open source software always appealed to me so this would be me
dipping my toes in the water to see what it's like.

Here's some links where you can find some more information on me:
\begin{itemize}
\setlength\itemsep{-0.1em}
\item \href{https://www.waltersmuts.com}{Personal Website - Contains some projects too}
\item \href{https://www.waltersmuts.com/Walter\%20Smuts\%20-\%20Curriculum\%20Vitae.pdf}{Curriculum Vitae}
\item \href{https://github.com/WalterSmuts}{Github}
\item \href{https://github.com/WalterSmuts/2022-summer-of-code-proposals}{The latex source for this proposal}
\end{itemize}


\end{normalsize}

\end{document}

