\documentclass{article}
\usepackage[utf8]{inputenc}
\usepackage[margin=0.75in]{geometry}
\usepackage{hyperref}
\usepackage{float}

\newcommand\ytl[2]{
\parbox[b]{10em}{\hfill{\bfseries\sffamily
#1}~$\cdots\cdots$~}\makebox[0pt][c]{$\bullet$}\vrule\quad \parbox[c]{8cm}{\vspace{3pt}\raggedright\sffamily #2.\\[3pt]}\\[-3pt]}

\begin{document}
\begin{center}

\LARGE{\textbf{Monado}} \\
\vspace{1em}
\Large{XR - Magic Window} \\
\vspace{1em}
\normalsize\textbf{Walter Smuts} \\
\normalsize{smuts.walter@gmail.com} \\
\vspace{1em}
\normalsize{Mentor: Jakob Bornecrantz} \\
\vspace{1em}
\normalsize{Google Summer of Code} \\

\end{center}
\begin{normalsize}

\section{Project Description:}

This project is one taken from the list of ideas presented by the organization
itself. Therefore the following description is taken verbatim from the
organizations website:

\begin{quote}

The core OpenXR 1.0 specification supports stereo VR headsets and monoscopic
handheld displays like smartphones or tablets which are supposed to be used as a
“magic window” into a VR world or for AR purposes; for this purpose the device’s
orientation and position is tracked to provide users the ability to move the
“window” view by moving the display device. A further use of monoscopic displays
is the “fish tank” configuration with a fixed display like a TV and instead the
head position of the user is tracked, to render the content behind the magic
window from the right perspective. (Example:
\url{https://www.youtube.com/watch?v=Jd3-eiid-Uw}). For this project, the student
will add support in Monado for tracking a face, figure out the relation of the
face/eyes to the a monitor and calculate fov values. The focus of this is not
making creating production ready code that in 100\% of the cases, but the
integration of the code into Monado. The small test application hello\_xr will
need changes to add better support for mono-scopic fish tank views, like improving
the scene setup. Depending on progress, the student can modify one or all of
Godot, Blender and Unreal to support mono-scopic fish tank mode.

\end{quote}

My understanding of this description is that the major work will lie in enabling
the face tracking feature in the Monado XR runtime. This will allow XR
applications to query the runtime for the user's head position and will enable use
cases like the magic window. The hello\_xr sample application can be used to test
this feature, although it might require some changes.

My intuition does however tell me the user experience may not be great. Most
consumer device cameras have a low framerate and high latency and may result in a
high end to end processing latency. I would expect the head position estimate to
lag behind the ground truth, and therefore the image on the screen will lag. This
could possibly be fixed by applying signal processing techniques such as a kalman
filter or incorporating gyroscopic and accelerometer data in the head position
estimator. If time permits, I'd like to explore these optimizations as well.

\section{Timeline:}

\begin{table}[H]
\centering
\begin{minipage}[t]{.7\linewidth}
\rule{\linewidth}{1pt}
\ytl{Week 1 - 2}{Download, compile, explore Monado runtime code and example applications}
\ytl{Week 3 - 4}{Get POC working, no matter how hacky}
\ytl{Week 5 - 6}{Compile a set of reviewable/maintainable patches}
\ytl{Week 7 - 8}{Make experience improvements - i.e. kalman filter, low pass filter or performance improvements}
\ytl{Week 9 - 10}{Code review process and merge PRs}
\ytl{Week 11 - 12}{Buffer Time - Things always take longer than expected}
\bigskip
\rule{\linewidth}{1pt}%
\end{minipage}%
\end{table}

\section{Deliverables:}

Lorem ipsum dolor sit amet, consectetur adipiscing elit, sed do eiusmod tempor
incididunt ut labore et dolore magna aliqua. Ut enim ad minim veniam, quis nostrud
exercitation ullamco laboris nisi ut aliquip ex ea commodo consequat. Duis aute
irure dolor in reprehenderit in voluptate velit esse cillum dolore eu fugiat nulla
pariatur. Excepteur sint occaecat cupidatat non proident, sunt in culpa pqui
officia deserunt mollit anim id est laborum.

\section{Contributor Info:}

This is in the separate tex file.


\end{normalsize}

\end{document}
