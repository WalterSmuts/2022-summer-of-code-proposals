\documentclass{article}
\usepackage[utf8]{inputenc}
\usepackage[margin=0.75in]{geometry}
\usepackage{hyperref}
\usepackage{float}

\hypersetup{urlcolor=blue,colorlinks=true,linktocpage}

\newcommand\ytl[2]{
\parbox[b]{10em}{\hfill{\bfseries\sffamily
#1}~$\cdots\cdots$~}\makebox[0pt][c]{$\bullet$}\vrule\quad \parbox[c]{8cm}{\vspace{3pt}\raggedright\sffamily #2.\\[3pt]}\\[-3pt]}

\begin{document}
\begin{center}


\LARGE{\textbf{QEMU}} \\
\vspace{1em}
\Large{Extend aarch64 support in rust-vmm/vmm-reference} \\
\vspace{1em}
\normalsize\textbf{Walter Smuts} \\
\normalsize{smuts.walter@gmail.com} \\
\vspace{1em}
\normalsize{Mentor: Andreea Florescu , Sergey Glushchenko} \\
\vspace{1em}
\normalsize{Google Summer of Code} \\

\end{center}
\begin{normalsize}

\section{Project Description:}

This project is one taken from the list of ideas presented by the organization
itself. Therefore the following description is taken verbatim from the
organization's website:

\begin{quote}

The vmm-reference is a reference implementation of a Rust VMM based on rust-vmm
crates. This is currently used for testing the integration of rust-vmm components,
with plans of extending it such that it becomes a starting point for custom Rust
VMMs.

The vmm-reference currently has support for x86\_64 and proof-of-concept level
support for aarch64. On aarch64, it just supports booting a dummy VM with no
devices, while on x86\_64 it has support for the virtio-net and virtio-blk devices.
The purpose of this project is to extend the existing functionality getting it
closer to what is already available on x86\_64, and consume the readily available
crates (for example vm-allocator) that would make the integration easier.

\end{quote}

My understanding of the description above is that the majority of the work would
involve writing rust code, reading ARM reference material, looking at the x86
counterpart, writing python integration tests and doing manual tests and
investigations. I think I'm in a good place take on this project since I've
previously worked on a team in Amazon that works on the EC2 Nitro Hypervisor,
writing Rust code.

\section{Timeline:}

The QEMU org did a really good job of breaking down the project already. Most of
the following is just taken directly from the org website:

\begin{table}[H]
\centering
\begin{minipage}[t]{.7\linewidth}
\rule{\linewidth}{1pt}
\ytl{Week 1 - 2}{Get a feel for the vmm-reference project and all the components}
\ytl{Week 3 - 4}{Set up interrupt controller}
\ytl{Week 5 - 6}{Add a real-time clock device}
\ytl{Week 7 - 8}{Add a serial port}
\ytl{Week 9 - 10}{Add flattened device tree (FDT) so guest has a machine description}
\ytl{Week 11 - 12}{Buffer Time - Things always take longer than expected}
\bigskip
\rule{\linewidth}{1pt}%
\end{minipage}%
\end{table}

\section{Deliverables:}

The overall deliverable is to flesh out aarch64 (ARM) support in vmm-reference to
make its features comparable to x86\_64. This can be broken up into these
sub-steps:

\begin{itemize}
\setlength\itemsep{-0.1em}
\item Add support for virtio block \& net for aarch64 aarch64
\item Allocate MMIO ranges \& GSIs aarch64
\item Create abstractions for initializing an FDT for booting aarch64
\item Setup the interrupt controller aarch64
\item vm-vcpu: Extend the KvmVm and KvmVcpu abstractions for aarch64 aarch64
\item Import system allocator (once it becomes available) aarch64
\item Integrate vm-vcpu changes for aarch64 support aarch64
\item Create a minimal FDT aarch64
\end{itemize}

\section{Contributor Info:}

I am Walter Smuts, an Electrical and Computer Engineering graduate from the
University of Cape Town. After I finished my studies I worked for Amazon Web
Services in Cape Town. I worked on a micro-services team on the EC2 control plane
for a year, mostly maintaining software written in Java. Thereafter I joined a
team doing new feature development on the AWS EC2 hypervisor working primarily in
C and Rust. I spent two years here and really enjoyed the work and my team.
Although I  was happy, I felt like I needed to do some more traveling while I'm
young before getting stuck in the corporate world.

I left my job with the idea to make some room to travel and think about my next
venture. Working on open source software always appealed to me so this would be me
dipping my toes in the water to see what it's like.

Here's some links where you can find some more information on me:
\begin{itemize}
\setlength\itemsep{-0.1em}
\item \href{https://www.waltersmuts.com}{Personal Website - Contains some projects too}
\item \href{https://www.waltersmuts.com/Walter\%20Smuts\%20-\%20Curriculum\%20Vitae.pdf}{Curriculum Vitae}
\item \href{https://github.com/WalterSmuts}{Github}
\item \href{https://github.com/WalterSmuts/2022-summer-of-code-proposals}{The latex source for this proposal}
\end{itemize}


\end{normalsize}

\end{document}

